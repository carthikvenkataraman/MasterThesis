\documentclass[MastersThesis.tex]{subfiles}
\begin{document}

\chapter{Introduction}
	Environmental issues such as global climate change and the risk of depleting the world's natural resources are continuing to press on with increasing relevance, prompting governments all over the world to work pro actively to address these issues. The recent implementation of the Euro 6 emission standards for heavy vehicles has been as strict as all previous standards with its scrutiny on the permissible levels of emissions. This prompted research into the feasibility and usefulness of considering longer vehicle combinations in lieu of the standard length of truck in Europe to reduce fuel consumption and hence emissions. The standard truck combination agreed to by the European Union allows for a maximum length of 18.75m and a maximum weight of 40 tons. However, based on feasibility studies an exception was made for certain member states, namely Finland, Denmark Sweden and the Netherlands. Sweden and Finland initially had allowed for longer trucks as per national regulations \cite{Davidsson11}. There was some friction caused upon their inclusion into the EU as they were unwilling to restrict themselves to the EU standard. This was driven by the fact that longer combinations had already been proven to be more economical, safer and even showed a reduction in fuel consumption \cite{SNRTRI08}. This lead to the cooperative signing of the European Modular System (EMS) agreement which allowed the aforementioned countries longer combinations on their roads provided they stuck to a set of standardised modules. According to the agreement the length of trucks permitted was extended to 25.25m and maximum weight now changed to 60 tons for Swedish roads.\cite{SNRTRI08}\\

	\begin{figure}[ht!]
		\begin{center}
			\includegraphics[width=0.88\textwidth, clip=true, trim=0 145 0 145]{figures/Introduction/fuelconsumption.pdf}
		\end{center}
		\caption{Fuel consumption reduction with longer vehicle combinations}
		\label{fig:fuelconsumptionred}
	\end{figure}

	The $Duo2$ project initiated by the Swedish government agency Vinnova in collaboration with Volvo AB, other companies and government agencies aims on building on the EMS configuration and taking it one step ahead. While the project focus is on the Swedish road industry, its results could lead to regulation changes that could potentially be extended to the other EMS nations and even the rest of EU in time. The goal of the project was to show that even longer combinations could reduce fuel consumption and thus $CO_2$ emissions by more than 25\%. Initial results of the project have been very positive as a reduction of 27\% percent was seen through testing a 32 m long combination weighing in at 80 tons through a testing on a set route from G\"oteborg to Malm\"o \cite{Cider13}

\section{The case for long combinations}

	Road freight transport constituted close to 73\% of all inland freight in the EU in the year 2010 \cite{RoadFreight2010}. Rich et al. predict a 43\% growth in freight transport by 2030 compared to the year 2005 \cite{RoadFreight20XX}. Most road freight today is carried out using combinations consisting of a tractor unit followed by an assortment of one short and one long trailer combination some of which are depicted in Figure \ref{CombinationsToday}.\\ 

	\begin{figure}[ht!]
		\begin{center}
			\includegraphics[width=0.3\textwidth]{figures/Introduction/CombinationsToday}
		\end{center}
		\caption{Truck and trailer combinations in use today}
		\label{CombinationsToday}
	\end{figure}

	Longer combinations consisting of more than three units are proposed to meet growing freight transport demands while reducing traffic and congestion-related losses, driver costs and overall operation costs, as seen in Figure \ref{ShortToLongCombinations}. Reduced fuel consumption and fixed costs also drive the productivity growth for long combinations. Nevertheless, long combinations with higher gross combination weights often suffer from reduced gradeability and tractability. This motivates the implementation of propelled trailer axles by means of hydraulic or electric power. Additional propulsion on trailer axles also provides increased lateral vehicle control possibilities.\\

	\begin{figure}[ht!]
		\begin{center}
			\includegraphics[width=\textwidth]{figures/Introduction/ShortToLongCombinations}
		\end{center}
		\caption{Replacing tractor-semitrailer combinations with the standard A-Double}
		\label{ShortToLongCombinations}
	\end{figure}

\section{Problem Domain}

	While it is established that additional propulsion aids improvement of combination gradeability and stability, configuration-specific questions such as the quantity of additional propulsion required for a given mission, payload and driving pattern, component sizing to achieve the same and energy management in consonance with the conventional tractor powertrain remain unanswered. \\

	\begin{figure}[ht!]
		\begin{center}
			\includegraphics[width=0.4\textwidth]{figures/Introduction/AdditionalPropulsion}
		\end{center}
		\caption{Adding propulsion on trailer axles improves combination gradeability}
		\label{AdditionalPropulsion}
	\end{figure}

	In this thesis, the problem of deriving the optimal propulsion configuration for a standard A-Double combination with additional electric propulsion on trailer axles is taken up. A few key configuration 'identifiers' or parameters are defined and an evaluation model developed to quantify the performance of the long vehicle combination with reference to a unique mission. Configuration identifiers sought to be addressed here include:

	\begin{itemize}
		\item Number and position of trailer axles propelled
		\item Buffer energy capacity on each trailing unit
		\item Electric motor rating for each propelled trailer axle
		\item Conventional tractor powertrain sizing
	\end{itemize}

	Thereafter, a custom optimisation algorithm that utilises the chosen concept evaluation model to derive the optimal propulsion configuration is constructed. The algorithm is then used to derive optimal configurations for several unique transport settings in different years. 

\section{Limitations}\label{sec:introlimitation}
	Any optimal solution arrived at is unique to the selection of the route and the definition of the favorable criteria. The vehicle model developed to represent the truck is a simple one-track longitudinal model with a rudimentary hybrid energy management controller. The design parameter space is limited and the efforts are focused on delivering a generic and modular tool to derive the optimal configuration that provides for easy addition and modification of design parameters and constants.

\end{document}
