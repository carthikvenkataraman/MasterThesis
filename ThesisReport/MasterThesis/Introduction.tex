\documentclass[MastersThesis.tex]{subfiles}
\begin{document}

\chapter{Introduction}
	Environmental issues such as global climate change and the risk of depleting the world's natural resources are continuing to press on with increasing relevance, prompting governments all over the world to work pro actively to address these issues. The recent implementation of the Euro 6 emission standards for heavy vehicles has been as strict as all previous standards with its scrutiny on the permissible levels of emissions. This prompted research into the feasibility and usefulness of considering longer vehicle combinations in lieu of the standard length of truck in Europe to reduce fuel consumption and hence emissions. The standard truck combination agreed to by the European Union allows for a maximum length of 18.75m and a maximum weight of 40 tons. However, based on feasibility studies an exception was made for certain member states, namely Finland, Denmark Sweden and the Netherlands. Sweden and Finland initially had allowed for longer trucks as per national regulations \cite{Davidsson11}. There was some friction caused upon their inclusion into the EU as they were unwilling to restrict themselves to the EU standard. This was driven by the fact that longer combinations had already been proven to be more economical, safer and even showed a reduction in fuel consumption \cite{SNRTRI08}. This lead to the cooperative signing of the European Modular System (EMS) agreement which allowed the aforementioned countries longer combinations on their roads provided they stuck to a set of standardised modules. According to the agreement the length of trucks permitted was extended to 25.25m and maximum weight now changed to 60 tons for Swedish roads.\cite{SNRTRI08}\\

	\begin{figure}[ht!]
		\begin{center}
			\includegraphics[width=0.88\textwidth, clip=true, trim=0 145 0 145]{figures/Introduction/fuelconsumption.pdf}
		\end{center}
		\caption{Fuel consumption reduction with longer vehicle combinations}
		\label{fig:fuelconsumptionred}
	\end{figure}

	The $Duo2$ project initiated by the Swedish government agency Vinnova in collaboration with Volvo AB, other companies and government agencies aims on building on the EMS configuration and taking it one step ahead. While the project focus is on the Swedish road industry, its results could lead to regulation changes that could potentially be extended to the other EMS nations and even the rest of EU in time. The goal of the project was to show that even longer combinations could reduce fuel consumption and thus $CO_2$ emissions by more than 25\%. Initial results of the project have been very positive as a reduction of 27\% percent was seen through testing a 32 m long combination weighing in at 80 tons through a testing on a set route from G\"oteborg to Malm\"o \cite{Cider13}

\section{Problem Statement} \label{sec:introproblemstatement}
	With encouragement from this end studies have now moved to focus other issues associated with increasing the length of the combination, such as maintaining stability and matching propulsion requirements. This increase may result in high speed amplification of the lateral motion from the first unit to the towed units which could be addressed through active steering of the towed units to ensure stability \cite{Kharrazi10} However this paper focuses on the longitudinal propulsion issue which arises from the fact that it is the same tractor module that is proposed by the EMS standard that has been used for the project up to this point. This means that longitudinal performance criteria such as gradeability and startability are affected by the increase in load on the engine as a result of increased weight. 
	This can be met by incorporating hybrid powertrain on the truck combinations through addition of motors on the axles to aid in propulsion requests. However any such addition of electrical components would have an associated cost with its installation. This may deter transport sector companies purchasing the trucks from going in for the electrified option as costs associated may outweigh the potential fuel saving benefits.\\

	Thus this paper aims at providing a method for the client to assess the level of hybridization if any that would be the most suitable for its particular requirements.

\section{Limitations}\label{sec:introlimitation}
	Any optimal solution arrived at would be unique to the selection of the route and the definition of the favorable criteria. Thus  robustness of the tool was more important than the final solution. The vehicle model developed to represent the truck does not fully capture all the effects of the powertrain with respect to losses, constraints between trailing units and machine properties. The paper will focus on the longitudinal dynamics of the hybrid powertrain as the vehicle model developed is a bicycle model and does not go into the analysing the lateral stability of proposed solutions. 

\end{document}
