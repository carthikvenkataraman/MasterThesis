Increasing volumes of road freight transport necessitate the development and use of fuel-efficient and economic transport vehicles and solutions. The transition from standard tractor-semitrailer combinations to longer truck and trailer combinations consisting of more than two units is in effect at the time of this research. Also, there is growing impetus for hybridisation and complete electrification of medium and heavy duty long haul trucks. Yet, the pertinent problem is the study of the economic and technical feasibility of such hybridisation for longer combinations where there can be several potential axles that can be electrically propelled. The problem is further compounded by the challenges and costs faced in using currently-expensive and heavy batteries and a variety of electric motors.\\

In this thesis, a generic physical modelling of long truck trailer combinations coupled with an energy management algorithm is developed. The energy management is made predictive in nature with prior knowledge of the mission route in order to optimise the use of the electric batteries. The propulsion is 'distributed' so as to allow each trailing unit to be independent, complete with its own battery and axle power distribution management.\\

Thereafter, an object-oriented programming architecture is constructed to implement the vehicle model. The A-Double combination is instantiated and validated against a physical test vehicle performing transport missions along Malm\"o and G\"oteborg in south-western Sweden.\\

Independently, a genetic algorithm is implemented on the same platform as the vehicle model. The genotype  structures corresponding to each long vehicle combination to be evaluated are developed and custom cross-over and mutation schemes are described and implemented. The algorithm is validated for standard benchmarking functions in two variables and for simple arithmetic functions involving the categorical variables representing the truck-trailer combinations.\\

Several objective functions for the long vehicle combinations are evaluated and a profit maximisation objective is chosen and mathematically formulated. The function parameters are varied to evaluate function sensitivity to each. The genetic algorithm is then used to derive the optimal axle propulsion configuration, battery size and motor rating for the mission specified above, with varying gross combination weights so as to maximise the chosen objective function. Trends in vehicle productivity are calculated over 15 following years from 2015 to 2030 and optimal long combination configurations are derived in intervals of 5 years. Principal contributing factors in each optimal configuration are identified and discussed.\\